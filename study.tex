\documentclass[ignorenonframetext]{beamer}
\usepackage{beamerthemeshadow}

%\documentclass{article}
%\usepackage{beamerarticle}
%\usepackage{graphicx}

\usepackage{lastpage}
\usepackage{xcolor}
\usepackage{pgf}
\usepackage{colortbl}
\usepackage{verbatim}
\usepackage{hyperref}

\newcommand{\bi}{\begin{itemize}}
\newcommand{\ei}{\end{itemize}}
\newcommand{\be}{\begin{enumerate}}
\newcommand{\ee}{\end{enumerate}}
\newcommand{\bd}{\begin{description}}
\newcommand{\ed}{\end{description}}
\newcommand{\prbf}[1]{\textbf{#1}}
\newcommand{\prit}[1]{\textit{#1}}
\newcommand{\beq}{\begin{equation}}
\newcommand{\eeq}{\end{equation}}
\newcommand{\bdm}{\begin{displaymath}}
\newcommand{\edm}{\end{displaymath}}

\newcommand{\ft}[1]{
  \frametitle{\begin{tabular}{p{4.2in}r} \textcolor{white}{#1} & \small{\insertframenumber / \inserttotalframenumber} \end{tabular}}
  %\frametitle{#1}
  \setbeamercovered{transparent=18}
}

\newcommand{\stepinv}{\setbeamercovered{invisible}}
\newcommand{\stopinv}{\setbeamercovered{transparent=18}}
\newcommand{\uncoverinv}[1]
{
  \setbeamercovered{invisible}
  \uncover<+->{#1}
  \setbeamercovered{transparent=18}
}
\newcommand{\ans}[1]{\textcolor{blue}{#1}}
\newcommand{\ansinv}[1]
{
  \setbeamercovered{invisible}
  \uncover<+->{\textcolor{blue}{#1}}
  \setbeamercovered{transparent=18}
}
\newcommand{\setinv}{\setbeamercovered{invisible}}
\newcommand{\setvis}{\setbeamercovered{transparent=18}}
\newcommand{\centerpic}[2]
{
  \begin{center}
  \includegraphics[#1]{#2}
  \end{center}
}
\newcommand{\h}[1]{\hat{#1}}
\newcommand{\ds}{\displaystyle}
\newcommand{\hl}[1]{\alt<#1>{\rowcolor{lightgreen}\hspace{-2.1pt}}{\hspace{-2.1pt}}}

\definecolor{mycolor}{rgb}{0.11,0.53,0.61}
\usecolortheme[named=mycolor]{structure}

\title{Smart Strength Training}
\author{ESS 230: Advanced Strength Training}
\date{}

\begin{document}

\frame{\titlepage\setcounter{framenumber}{0}}

\frame
{
  \ft{Smart Strength Training Practices}
  \bi
  \item<+-> Regular training: sessions every week, multiple times per week.
  \item<+-> Alternate weights: Exercise a number of different muscles in a session, don't spend too much time with one exercise.
  \item<+-> Alternate types of exercise during week: on off days, work on cardio exercises.
  \item<+-> Do not let weight lifting sessions go too long.
  \item<+-> Don't just look at the weights, that won't do anything.
  \item<+-> Don't cram, that will hurt you.
  \ei 
}


\frame
{
  \ft{Smart Study Practices}
  \bi
  \item<+-> Regular sessions: study every week.
    \bi
    \item<+-> This is more productive (takes less total studying time).
    \item<.-> Leads to long-term memory (you'll remember for the final, next class).
    \ei
  \item<+-> Alternate topics and types of exercises
    \bi
    \item<+-> Don't spend a whole session on one topic.
    \item<.-> Don't spend whole session on one method, change it up (review notes, practice problems, test yourself, etc).
    \ei
  \item<+-> Alternate courses during week.
  \item<+-> Don't let study sessions go on too long.
  \item<+-> Don't just look at your materials (i.e. only review notes is next to worthless).
  \item<+-> Don't cram, it will hurt you in the long run.
  \ei
}

\frame{
  \ft{Other Smart Practices}
  \bi
  \item<+-> If reviewing notes or textbook, only do it briefly.
    \bi
    \item<+-> \textit{Recognition Fallacy}
    \ei
  \item<+-> Test yourself: take practice tests, make or find your own practice questions.
  \item<+-> Take notes from your book, or take notes from your notes.
  \item<+-> Don't study isolated facts. 
    \bi
    \item<+-> Don't focus on terms, highlighted material, etc.
    \item<+-> This leads to little comprehension.
    \ei
  \item<+-> When to best review notes: \textit{Right after class!!  Do it within hours, same day!!}
  \item<+-> Change location.
  \item<+-> Do not ``multitask.''  Human brain is not efficient at changing tasks.
    \bi
    \item<+-> No TV, radio, computer, text messaging, facebook, etc.
    \ei
  \item<+-> Postpone pleasurable activities.
  \ei
}

\frame
{
  \ft{Watch These Videos!}
  \begin{center}
    Dr. Stephen Chew, a Cognitive Psychologist at Samford University explains to college students how the mind works (and doesn't work) to effectively learn.\\
    \vspace*{0.2in}
    \textcolor{blue}{\href{http://www.youtube.com/playlist?list=PL85708E6EA236E3DB}{http://www.youtube.com/playlist?list=PL85708E6EA236E3DB}}
  \end{center}
}



\end{document}
